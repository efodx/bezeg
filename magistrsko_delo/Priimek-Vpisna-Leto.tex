%\documentclass[mat2, tisk]{fmfdelo}
% \documentclass[fin2, tisk]{fmfdelo}
\documentclass[isrm2, tisk]{fmfdelo}


% - ime datoteke z viri (vključno s končnico .bib), če uporabljate BibTeX


% \documentclass[ped, tisk]{fmfdelo}
% Če pobrišete možnost tisk, bodo povezave obarvane,
% na začetku pa ne bo praznih strani po naslovu, …

%%%%%%%%%%%%%%%%%%%%%%%%%%%%%%%%%%%%%%%%%%%%%%%%%%%%%%%%%%%%%%%%%%%%%%%%%%%%%%%
% METAPODATKI
%%%%%%%%%%%%%%%%%%%%%%%%%%%%%%%%%%%%%%%%%%%%%%%%%%%%%%%%%%%%%%%%%%%%%%%%%%%%%%%

% - vaše ime
\avtor{Kevin Štampar}

% - naslov dela v slovenščini
\naslov{Orodje za uvod v Bezierjeve krivulje}

% - naslov dela v angleščini
\title{Tool for introduction into Bezier curves}

% - ime mentorja/mentorice s polnim nazivom:
%   - doc.~dr.~Ime Priimek
%   - izr.~prof.~dr.~Ime Priimek
%   - prof.~dr.~Ime Priimek
%   za druge variante uporabite ustrezne ukaze
\mentor{prof.~dr.~Emil Žagar}
% \somentor{...}
% \mentorica{...}
% \somentorica{...}
% \mentorja{...}{...}
% \somentorja{...}{...}
% \mentorici{...}{...}
% \somentorici{...}{...}

% - leto magisterija
\letnica{2024}

% - povzetek v slovenščini
%   V povzetku na kratko opišite vsebinske rezultate dela. Sem ne sodi razlaga
%   organizacije dela, torej v katerem razdelku je kaj, pač pa le opis vsebine.
\povzetek{Tukaj napišemo povzetek vsebine. Sem sodi razlaga vsebine in ne opis tega, kako je delo organizirano.}

% - povzetek v angleščini
\abstract{An abstract of the work is written here. This includes a short description of
the content and not the structure of your work.}

% - klasifikacijske oznake, ločene z vejicami
%   Oznake, ki opisujejo področje dela, so dostopne na strani https://www.ams.org/msc/
\klasifikacija{74B05, 65N99}

% - ključne besede, ki nastopajo v delu, ločene s \sep
\kljucnebesede{integracija\sep kompleks}

% - angleški prevod ključnih besed
\keywords{integration\sep complex} % angleški prevod ključnih besed

% - neobvezna zahvala
\zahvala{
    Zahvaljujem se mentorju za zelo sproščen odnos!
}

% - program dela, ki ga napiše mentor z osnovno literaturo
\programdela{
    Mentor naj napiše program dela skupaj z osnovno literaturo.
}

\osnovnaliteratura{
% Literatura mora biti tukaj posebej samostojno navedena (po pomembnosti) in ne
% le citirana. V tem razdelku literature ne oštevilčimo po svoje, ampak uporabljamo
% ukaz \vnosliterature, v katerega vpišemo citat
    \vnosliterature{lebedev2009introduction}
    \vnosliterature{gurtin1982introduction}
    \vnosliterature{zienkiewicz2000finite}
    \vnosliterature{STtemplate}
}

% - ime datoteke z viri (vključno s končnico .bib), če uporabljate BibTeX
\literatura{literatura.bib}

%%%%%%%%%%%%%%%%%%%%%%%%%%%%%%%%%%%%%%%%%%%%%%%%%%%%%%%%%%%%%%%%%%%%%%%%%%%%%%%
% DODATNE DEFINICIJE
%%%%%%%%%%%%%%%%%%%%%%%%%%%%%%%%%%%%%%%%%%%%%%%%%%%%%%%%%%%%%%%%%%%%%%%%%%%%%%%

% naložite dodatne pakete, ki jih potrebujete
\usepackage{units}        % fizikalne enote kot \unit[12]{kg} s polovico nedeljivega presledka, glej primer v kodi
\usepackage{graphicx}     % za slike
% \usepackage{tikz}
% VEČ ZANIMIVIH PAKETOV
% \usepackage{array}      % več možnosti za tabele
% \usepackage[list=true,listformat=simple]{subcaption}  % več kot ena slika na figure, omogoči slika 1a, slika 1b
% \usepackage[all]{xy}    % diagrami
% \usepackage{doi}        % za clickable DOI entrye v bibliografiji
%\usepackage{enumitem}     % več možnosti za sezname

% Za barvanje source kode
% \usepackage{minted}
% \renewcommand\listingscaption{Program}

% Za pisanje psevdokode
\usepackage{algpseudocode}  % za psevdokodo
\usepackage{algorithm}
\usepackage{algorithmicx}
\usepackage{mathtools}
\usepackage{hyperref}
\floatname{algorithm}{Algoritem}
\renewcommand{\listalgorithmname}{Kazalo algoritmov}

% deklarirajte vse matematične operatorje, da jih bo LaTeX pravilno stavil
% \DeclareMathOperator{\...}{...}

% vstavite svoje definicije ...
\newcommand{\R}{\mathbb R}
\newcommand{\N}{\mathbb N}
\newcommand{\Z}{\mathbb Z}
% Lahko se zgodi, da je ukaz \C definiral že paket hyperref,
% zato dobite napako: Command \C already defined.
% V tem primeru namesto ukaza \newcommand uporabite \renewcommand
\newcommand{\C}{\mathbb C}
\newcommand{\Q}{\mathbb Q}
\newcommand{\Pn}{\mathbb P_n}
\newcommand{\p}{\textbf{p}}


%Bernstein
\newcommand{\bernsteinbase}[3]{\binom{#1}{#2}t^{#1}(1-t)^{#2}}
\newcommand{\bernstein}[2]{\binom{#1}{#2}t^{#2}(1-t)^{#1-#2}}
\newcommand{\lilb}[2]{b_{#1,#2}(t)}
\newcommand{\bigb}[1]{B_{#1}(t)}
\newcommand{\bigbb}[1]{\textbf{B}_{#1}(t)}
\newcommand{\bigbbod}[2]{\textbf{B}_{#1}(#2)}
\newcommand{\bigbbt}{\textbf{B}(t)}
\newcommand{\bigbo}[1]{B'_{#1}(t)}
\newcommand{\bernsteinsum}[2]{\sum_{#1=0}^{#2} \beta_{#1}\lilb{#1}{#2}}
\newcommand{\bernsteinsump}[2]{\sum_{#1=0}^{#2} \p_{#1}\lilb{#1}{#2}}
\newcommand{\bernsteinsumtri}[3]{\sum_{#1=0}^{#2} #3_{#1}\lilb{#1}{#2}}
\newcommand{\bernsteinsumtritri}[3]{\sum_{#1=0}^{#2} #3\lilb{#1}{#2}}
\newcommand{\bernsteinsumtridva}[2]{\sum_{#1=0}^{#2} \lilb{#1}{#2}}

\newcommand{\bsum}{\bernsteinsum{i}{n}}
%%%%%%%%%%%%%%%%%%%%%%%%%%%%%%%%%%%%%%%%%%%%%%%%%%%%%%%%%%%%%%%%%%%%%%%%%%%%%%%
% ZAČETEK VSEBINE
%%%%%%%%%%%%%%%%%%%%%%%%%%%%%%%%%%%%%%%%%%%%%%%%%%%%%%%%%%%%%%%%%%%%%%%%%%%%%%%

\begin{document}

    \section{Uvod}
    Napišite kratek zgodovinski in matematični uvod. Pojasnite motivacijo za problem, kje
    nastopa, kje vse je bil obravnavan. Na koncu opišite tudi organizacijo dela -- kaj je v
    katerem razdelku.


    \section{Bezierjeve krivulje}\label{sec:bezierjeve-krivulje}
    V tem razdelku bomo predstavili osnove Bezierjevih krivulj.
    Začeli bomo z Bernsteinovimi polinomi, ki jih bomo uporabili pri definiciji Bezierjevih krivulj.
    Predstavili bomo Decasteljaujev algoritem, ki je ključen za stabilen način računanja točk Bezierjevih krivulj.
    Nadaljevali pa bomo z nekaj metodami na Bezierjevih krivuljah, ki so ključne za njihovo rabo v računalniško podprtem grafičnem oblikovanju.

    \subsection{Bernsteinovi polinomi}\label{subsec:bernsteinovi-polinomi}
    Bernsteinove polinome je najprej uporabil Sergei Bernstein pri dokazu Weierstrassovega izreka.
    Kasneje jih je Pierre Bezier uporabil pri definiciji Bezierjeve krivulje.
    %izumil za rabo RPGO!!
    V tem razdelku bomo predstavili nekaj njihovih osnovnih lastnosti, ki so ključne za delovanje Bezierjevih krivulj.
    $i$-ti \textit{Bernsteinov bazni polinom} stopnje $n$ definiramo kot $\lilb{i}{n} \coloneqq\bernstein{n}{i}$.
    Linearni kombinaciji takšnih polinomov t.j.\ $\bigb{n} \coloneqq \bsum$, pravimo \textit{Bernsteinov polinom} stopnje $n$.
    V izreku\ref{izrek:bernsteinovi_lastnosti} naštejemo nekaj lastnosti Bernsteinovih polinomov.
%    Takšni polinomi so bili prvič uporabljeni v konstruktivnem(????) dokazu Weierstrassovega aproksimacijskega izreka.

    \begin{izrek}{Lastnosti Bernsteinovih polinomov}
        \label{izrek:bernsteinovi_lastnosti}

        \begin{enumerate}
            \item $\lilb{i}{n} = 0$ za $i<0$ ali $i>n$, interpolacija končnih točk \label{izrek:bernsteinovi_lastnosti:interpolacija}
            \item $\lilb{i}{n} \geq 0$ za $t\in[0,1]$, pozitivnost \label{izrek:bernsteinovi_lastnosti:pozitivnost}
            \item $b_{i,n}(0) = \delta_{i,0} \quad \text{in} \quad  b_{i,n}(1) = \delta_{i,n}, \text{ kjer je }  \delta_{i,j} = \begin{cases}
                                                                                                                                    1 & i=j \\
                                                                                                                                    0 & 1\neq j
            \end{cases}$
            \item $b_{i,n}(1-t) = \lilb{n-i}{n}$, simetrija \label{izrek:bernsteinovi_lastnosti:simetrija}
            \item $\bernsteinsumtridva{i}{n} = 1$, razčlenitev enote \label{izrek:bernsteinovi_lastnosti:enota}
            \item $\lilb{n}{i} = (1-t)\lilb{n-1}{i} + t\lilb{n-1}{i-1}$ \label{izrek:bernsteinovi_lastnosti:rekruzija}
            \item $b'_{i,n}(t)=n(\lilb{i-1}{n-1} - \lilb{i}{n-1})$ in  $\bigbo{n}=n\sum^{n-1}_{i=0}(\beta_{i+1}-\beta_{i})b_{i,n-1}(t)$ \label{izrek:bernsteinovi_lastnosti:odvod}
        \end{enumerate}
    \end{izrek}
    \begin{dokaz}
        Točki (1) in (2) očitno izhajata iz lastnosti binomskega simbola.
        Dokažimo ostale.

        \noindent(3) Namesto spremenljivke $t$ v enačbo za bernsteinov bazni polinom $\lilb{i}{n}$ vstavimo izraz $1-t$ in uporabimo lastnost binomskega simbola $\binom{n}{i} = \binom{n}{n-i}$, dobimo

        \[b_{i,n}(1-t) = \binom{n}{i}(1-t)^i(1-(1-t))^{n-i} =  \binom{n}{n-i}(1-t)^it^{n-i} = b_{n-i,i}(t).\]

        \noindent(4) Za $1 = 1^n = (1-t+t)^n = ((1-t) + t)^n$ uporabimo binomski izrek, dobimo
        \[\left((1-t) + t\right)^n = \sum_{i=0}^{n}\bernstein{n}{i} = \sum_{i=0}^n \lilb{i}{n}.\]

        \noindent(5) Uporabili bomo lastnost binomskega simbola $ \binom{n-1}{i} + \binom{n-1}{i-1} = \binom{n}{i}.$
        \begin{align}
            &(1-t)\lilb{n-1}{i} + t\lilb{n-1}{i-1} = \nonumber \\
            &= (1-t)\binom{n-1}{i}t^{i}(1-t)^{n-i-1} + t\binom{n-1}{i-1}t^{i-1}(1-t)^{n-i} &= \nonumber \\
            &= \binom{n-1}{i}t^{i}(1-t)^{n-i} + \binom{n-1}{i-1}t^{i}(1-t)^{n-i} &= \nonumber \\
            &= \binom{n}{i}t^{i}(1-t)^{n-i} &= \nonumber \\
            &= \lilb{n}{i}
        \end{align}

        \noindent(6) Dodaj dokaz!
        Lahko tudi vecdimenzionalno ane
    \end{dokaz}

    \subsection{Večdimenzionalne oznake}
    Z željo po krajših, bolj preglednih zapisih, bomo uvedli večdimenzionalne oznake.
    Večdimenzionalnost bomo ponazarjali z odebelitvijo črke.
    Tako bomo večdimenzionalne točke označili z $\mathbf{x}=(x_0,x_1,\dots,x_n)$, večdimenzionalne funkcije $f:\R\to\R^{n+1}$ pa z $\mathbf{f}(x)=\left( f_0(x),f_1(x),\dots,f_n(x) \right)$.

    \subsection{Bezierjeve krivulje}
    Če v Bernsteinov polinom stopnje $n$ namesto realnega števila $\beta_i$ vstavimo točke $\p_i\in\R^2$, dobimo t.i.\ \textit{Bezierjevo krivuljo} stopnje $n$  t.j.\ $\bigbb{n}=\bernsteinsump{i}{n}$.
    Točkam $\p_i$ pravimo \textit{kontrolne točke}, poligonu, ki ga dobimo, če povežemo točke $\p_i$ in $\p_{i+1}$ za $i=[0,\dots,n-1]$, ter točki $\p_0$ in $\p_n$ pa \textit{kontrolni poligon}.


    \begin{izrek}{Lastnosti Bezierjevih krivulj}
        \begin{enumerate}
            \item $\bigbbod{n}{0}=\p_0$ in $\bigbbod{n}{1}=\p_n$, interpolacija končnih točk
            \item $\phi(\bernsteinsump{i}{n}) =\bernsteinsumtritri{i}{n}{\phi(\p_i)}$, afina invarianca
            \item Krivulja leži znotraj konveksne ovojnice svojih kontrolnih točk.
        \end{enumerate}
    \end{izrek}
    Preden izrek dokažemo, povejmo zakaj so zgornje lastnosti pomembne za potrebe grafičnega oblikovanja.
    Interpolacija končnih točk in lastnost (3) sta pomembni, saj omogočajo enostavno kontrolo krivulje.
    Afina invarianca pa je pomembna za enostavno premikanje, rotiranje itd.\ krivulje, saj lahko krivuljo transfomiramo tako, da transformacijo uporabimo na kontrolnih točkah.


    \begin{dokaz}
        ~\\
        \noindent (1) $\bigbbod{n}{0}=\sum_{i=0}^{n}\p_{i}b_{n,i}(0) = \sum_{i=0}^{n}\p_{i}\delta_{0,i} = \p_0$.
        Enako lahko naredimo tudi za $\bigbbod{n}{1}.$

        \noindent (2) Naj bo $\phi$ afina preslikava, velja torej $\phi(x) = A\mathbf{x} + \mathbf{b}.$
        \begin{align*}
            \phi\left(\sum_{i=0}^{n}\p_{i}b_{n,i}(t)\right) &= A\left(\sum_{i=0}^{n}\p_{i}b_{n,i}(t)\right) + \mathbf{b} &&=  \sum_{i=0}^{n}A\p_{i}b_{n,i}(t) + \mathbf{b}  \\
            &= \sum_{i=0}^{n}A\p_{i}b_{n,i}(t) + \sum_{i=0}^{n}\mathbf{b}b_{n,i}(t) &&= \sum_{i=0}^{n}(A\p_{i}+\mathbf{b})b_{n,i}(t) \\
            &= \bernsteinsumtritri{i}{n}{\phi(\p_i)}
        \end{align*}
        \noindent (3) Konveksna ovojnica kontrolnih točk Bezierjeve krivulje je množica vseh konveksnih kombinacij teh točk t.j.\ $\sum_{i=0}^{n}\lambda_i\p_{i}$, kjer so $\lambda_i$ pozitivna realna števila za katere velja $\lambda_0 + \lambda_1 + \dots + \lambda_n = 1$.
        Ker so Bernsteinovi polinomi za poljuben $t\in[0,1]$ razčlenitev enote in velja $\lilb{n}{i}\geq0$, lahko zapišemo $\lambda_i=\lilb{n}{i}$.
    \end{dokaz}

    \subsection{Decasteljau}
    Računanje Bernsteinovih polinomov direktno preko njihovih enačb je precej nestabilno****, za rabo v računalništvu pa je pomembno, da uporabljamo stabilne metode računanja.
    S pomočjo Decasteljaujevega algoritma lahko računamo točke Bezierjevih krivulj stabilno****, zanj pa potrebujemo naslednji izrek.

    \begin{izrek}
        Naj bo $\bigbbt_{[\p_0,\p_1,\dots,\p_n]}$ Bezierjeva krivulja $n$-te stopnje s kontronlimi točkami $\p_0,\p_1,\dots,\p_n$.
        Potem lahko njene točke rekurzivno računamo s pomočjo naslednjega izraza \[\bigbbt_{[\p_0,\p_1,\dots,\p_n]} = (1-t)\bigbbt_{[\p_0,\p_1,\dots,\p_{n-1}]} +t\bigbbt_{[\p_1,\dots,\p_n]}.\]
    \end{izrek}

    Izrek tudi dokažimo.

    \begin{dokaz}
        \begin{align*}
        (1-t)
            &
            \bigbbt_{[\p_0,\p_1,\dots,\p_{n-1}]}+t\bigbbt_{[\p_1,\dots,\p_n]} = \\
            &= (1-t)\bernsteinsump{i}{n-1}+t\sum_{i=0}^{n-1} \p_{i+1}\lilb{n-1}{i} \\
            &= (1-t)\bernsteinsump{i}{n-1}+ t\sum_{i=1}^{n} \p_{i}\lilb{n-1}{i-1} \\
            &= \p_0(1-t)\lilb{0}{n-1} + \sum_{i=1}^{n-1}\p_{i}(1-t)\lilb{n-1}{i} +  \sum_{i=1}^{n-1} \p_{i}t\lilb{n-1}{i-1} + \p_n \lilb{n-1}{n-1} \\
            &= \p_0(1-t)\lilb{0}{n-1} + \sum_{i=1}^{n-1}\left((1-t)\lilb{n-1}{i} + t\lilb{n-1}{i-1}\right)\p_{i} + \p_n \lilb{n-1}{n-1} \\
            &= \p_0\lilb{n}{0} + \sum_{i=1}^{n-1}\p_{i}\lilb{n}{i} + \p_n \lilb{n}{n} \\
            &= \sum_{i=0}^{n}\p_{i}\lilb{n}{i} \\
        \end{align*}
    \end{dokaz}
    \begin{algorithm}
        \caption{Decasteljau}
        \begin{algorithmic}
            \State $\p \gets \p_0,\p_1,\dots,\p_n$
            \For{$i = 0,1,\dots n$}
                \State $\p_i^0(t)=\p_i$
            \EndFor
            \For{$r = 1,2,\dots n$}
                \For{$i=0,1,\dots,n-r$}
                    \State $\p_i^r(t)=(1-t)\p_i^{r-1}(t)+t\p_{i+1}^{r-1}(t)$
                \EndFor
            \EndFor
            \State \Return $\p_0^n$
        \end{algorithmic}
    \end{algorithm}

    Decasteljaujev algoritem ima tudi geometrijski pomen.
    Predstavlja namreč nekakšno zaporedno interpolacijo točk.

    \subsection{Metode Bezierjevih krivulj}
    V tem razdelku bomo predstavili nekaj metod Bezierjevih krivulj, ki so uporabne pri grafičnem oblikovanju.

    \subsubsection{Subdivizija}
    Recimo, da smo se kot računalniški grafik znašli v situaciji, ko bi želeli obdržati le en kos Bezierjeve krivulje.
    Naj bo to kos krivulje, ki ga dobimo tako, da za prvotno krivuljo omejimo parameter $t$ na interval $[0,t_0]$ za neko fiksno realno število $t_0<1$.
    Če se sedaj za trenutek vrnemo nazaj k
    \begin{figure}
        \centering
        \includegraphics{images/subdivizija}
        \caption{Subdivizija}
        \label{fig:subdivizija}
    \end{figure}

    \subsubsection{Ekstrapolacija}
    Motivacija: Zelimo podaljsati krivuljo

    \subsubsection{Dvig stopnje}
    Motivacija: Zelimo primerjati dve krivulji, pa sta razlicne stopnje.

    \subsection{Racionalne Bezierjeve krivulje}


    $(w(t),x(t),y(t) => \left(1,\frac{x(t)}{w(t)},\frac{y(t)}{w(t)})\right)$

    \subsubsection{Metode racionalnih Bezierjevih krivulj}
    Metode Bezierjevih krivulj se zlahka razširijo na racionalne Bezierjeve krivulje tako, da racionalno Bezierjevo krivuljo ($\in\R^2$) preslikamo v Bezierjevo krivuljo reda $\in\R^2$, na njej izvedemo metodo, nato pa jo preslikamo nazaj v racionalno Bezierjevo krivuljo. Slednje ni najbolj stabilno, zato v praksi uporabimo nekoliko bolj stabilne načine računanja. Načine bomo le podali, ne bomo jih pa tudi dokazovali.

    \subsubsection{Decasteljau}

    \subsubsection{Subdivizija}

    \subsubsection{Ekstrapolacija}

    \subsubsection{Dvig stopnje}


    \section{Zlepki (Bezierjevih krivulj)}
    Motivacija: za vsak n imamo n**2 racunanja pri bezierjevih krivulah. Radi bi manj racunanja pa se vseeno obdrzali cimvecjo natancnost. Pridejo na pomoc zlepki- lokalno bomo ohranili natancnost a racunat ne bomo rabli dosti!!
    V tem razdelku, se bomo posvečali zlepkom Bezierjevih krivulj.

    \begin{definicija}
        Zlepek $s:[a,b]\to \R $  stopnje $n$ nad zaporedjem stičnih točk \[a=u_0 < u\_1 < \cdots < u_{m-1} < u_m = b\]
        je odsekoma polinomska funkcija, za katero velja $s|_{[u_{l-1},u_l]} \in \Pn$.
    \end{definicija}

    \subsection{C0}

    \subsection{C1}

    \subsection{C2}

    \subsection{G1}

    \subsection{Alfa parametrizacije}


    \section{PH Krivulje}

    \subsection{Racionalna dolžina krivulje}

    \subsection{Racionalni odmik krivulje}

    \subsection{Enakomerna parametrizacija}


    \section{Orodje za uvod v Bezierjeve krivulje - Bezeg}
    Vsi koncepti predstavljeni v magistrskem delu so tudi implementirani na spletni strani.
    Za graf sem uporabil jsxgraph. Za oblikovanje bootstrap. Za ogrodje pa React.

    \subsection{Implementacija konceptov magistrskega dela}






    \newpage



    \newpage
    \newpage
    \newpage


    \section{Integrali po \texorpdfstring{$\omega$}{ω}-kompleksih}

    \subsection{Definicija}
    \begin{definicija}
        Neskončno zaporedje kompleksnih števil, označeno z $\omega = (\omega_1, \omega_2, \ldots)$,
        se imenuje \emph{$\omega$-kompleks}.\footnote{To ime je izmišljeno.}

        Črni blok zgoraj je tam namenoma. Označuje, da \LaTeX{} ni znal vrstice prelomiti pravilno
        in vas na to opozarja. Preoblikujte stavek ali mu pomagajte deliti problematično besedo z
        ukazom \verb|\hyphenation{an-ti-ko-mu-ta-ti-ven}| v preambuli.
    \end{definicija}
    \begin{trditev}[Znano ime ali avtor]
        \label{trd:obstoj-omega}
        Obstaja vsaj en $\omega$-kompleks.
    \end{trditev}
    \begin{proof}
        Naštejmo nekaj primerov:
        \begin{align}
            \omega &= (0, 0, 0,& &= \dots), \label{eq:zero-kompleks} \\
            \omega &= (1, i, & &=-1, -i, 1, \ldots), \nonumber \\
            \omega &= (0,& &= 1, 2, 3, \ldots). \nonumber \qedhere  % postavi QED na zadnjo vrstico enačbe
        \end{align}
    \end{proof}


    \section{Tehnični napotki za pisanje}

    \subsection{Sklicevanje in citiranje}
    Za sklice uporabljamo \verb|\ref|, za sklice na enačbe \verb|\eqref|, za citate \verb|\cite|. Pri
    sklicevanju in citiranju sklicano številko povežemo s prejšnjo besedo z nedeljivim presledkom
    $\sim$, kot npr.\ \verb|iz trditve~\ref{trd:obstoj-omega} vidimo|.

    \begin{primer}
        Zaporedje~\eqref{eq:zero-kompleks} iz dokaza trditve~\ref{trd:obstoj-omega} na
        strani~\pageref{trd:obstoj-omega} lahko najdemo tudi v Spletni enciklopediji zaporedij~\cite{oeis}.
        Citiramo lahko tudi bolj natančno~\cite[trditev 2.1, str.\ 23]{lebedev2009introduction}.
    \end{primer}

    \subsection{Okrajšave}
    Pri uporabi okrajšav \LaTeX{} za piko vstavi predolg presledek, kot npr. tukaj. Zato se za vsako
    piko, ki ni konec stavka doda presledek običajne širine z ukazom \verb*|\ |, kot npr.\ tukaj.
    Primerjaj z okrajšavo zgoraj za razliko.

    \subsection{Vstavljanje slik}
    Sliko vstavimo v plavajočem okolju \texttt{figure}. Plavajoča okolja \emph{plavajo} po tekstu, in
    jih lahko postavimo na vrh strani z opcijskim parametrom `\texttt{t}', na lokacijo, kjer je v kodi s
    `\texttt{h}', in če to ne deluje, potem pa lahko rečete \LaTeX u, da ga \emph{res} želite tukaj,
    kjer ste napisali, s `\texttt{h!}'. Lepo je da so vstavljene slike vektorske (recimo \texttt{.pdf}
    ali \texttt{.eps} ali \texttt{.svg}) ali pa \texttt{.png} visoke resolucije (več kot
    \unit[300]{dpi}). Pod vsako sliko je napis in na vsako sliko se skličemo v besedilu. Primer
    vektorske slike je na sliki~\ref{fig:sample}. Vektorsko sliko prepoznate tako, da močno
    zoomate v sliko, in še vedno ostane gladka. Več informacij je na voljo na
    \url{https://en.wikibooks.org/wiki/LaTeX/Floats,_Figures_and_Captions}. Če so slike bitne, kot na
    primer slika~\ref{fig:image}, poskrbite, da so v dovolj visoki resoluciji.

    \begin{figure}[h]
        \centering
        \includegraphics[width=0.6\textwidth]{images/sample.pdf}
% \caption[caption za v kazalo]{Dolg caption pod sliko}
        \caption[Primer vektorske slike.]{Primer vektorske slike z oznakami v enaki pisavi, kot jo
        uporablja \LaTeX{}. Narejena je s programom Inkscape, \LaTeX{} oznake so importane v
        Inkscape iz pomožnega PDF.}
        \label{fig:sample}
    \end{figure}

    \begin{figure}[h]
        \centering
        \includegraphics[width=0.8\textwidth]{images/image.png}
        \caption[Primer bitne slike.]{Primer bitne slike, izvožene iz Matlaba. Poskrbite, da so slike v
        dovolj visoki resoluciji in da ne vsebujejo prosojnih elementov (to zahteva PDF/A-1b format).}
        \label{fig:image}
    \end{figure}

    \subsection{Kako narediti stvarno kazalo}
    Dodate ukaze \verb|\index{polje}| na besede, kjer je pojavijo, kot tukaj\index{tukaj}.
    Več o stvarnih kazalih je na voljo na \url{https://en.wikibooks.org/wiki/LaTeX/Indexing}.

    \subsection{Navajanje literature}
    Članke citiramo z uporabo \verb|\cite{label}|, \verb|\cite[text]{label}| ali pa več naenkrat s
    \verb|\cite\{label1, label2}|. Tudi tukaj predhodno besedo in citat povežemo z nedeljivim presledkom
    $\sim$. Na primer~\cite{chen2006meshless,liu2001point}, ali pa \cite{kibriya2007empirical},\cite{kibriya2007empirical}, ali pa
    \cite[str.\ 12]{trobec2015parallel}, \cite[enačba (2.3)]{pereira2016convergence}.
    Vnosi iz \verb|.bib| datoteke, ki niso citirani, se ne prikažejo v seznamu literature, zato jih
    tukaj citiram.~\cite{vene2000categorical}, \cite{gregoric2017stopniceni}, \cite{slak2015induktivni},
    \cite{nsphere}, \cite{kearsley1975linearly}, \cite{STtemplate}, \cite{NunbergerTand}, \cite{vanoosten2008realizability}.

\end{document}
